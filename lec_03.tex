\lecture{3}{17/10/2024}{Critical Exponents}

If one fixes $T = T_C$ we see
\begin{align}
    f \approx - Bm + \frac{1}{12} T_C m^{4} + \cdots
,\end{align}
where minimizing leads to $m^3 \sim B \implies m \sim  B^{\frac{1}{3}}$ for small $m$.

It is natural then to define $\pdv{m}{B}$.

\begin{definition}
    We call $\xi$ the \textbf{magnetic susceptibility} and is given by
    \begin{align}
        \xi = \left( \pdv{m}{B} \right)_T
    .\end{align}
\end{definition}

For $T > T_C$, we see
\begin{align}
    f \left( m \right) \approx - B m + \frac{1}{2} \left( T - T_C \right) m^2 + \cdots
,\end{align}
which implies
\begin{align}
    m \approx \frac{B}{T - T_C} \implies \xi = \frac{1}{T - T_C}
.\end{align}

For $T < T_C$, we write $m = m_0 + \delta m$ and solve for $\delta m$ to leading order in $B$. This leads to
\begin{align}
    m = m_0 + \frac{B}{2 \left( T_C - T \right) }
,\end{align}
which implies
\begin{align}
    \xi \approx \frac{1}{2 \left( T_C - T \right) }
,\end{align}
and thus
\begin{align}
    \xi \sim  \frac{1}{\left| T - T_C \right| }
,\end{align}
which diverges at $T_C$ from both sides.

\subsection{Validity of Mean Field Theory}

Naturally, given all these approximations, one may ask if they, especially mean field theory, give correct results. This is a function of dimension it turns out.
\begin{itemize}
    \item In $d = 1$, MFT is not valid as there are no phase transitions.
    \item In $d = 2,3$, the phase diagram is qualitatively correct, but qualitative predictions at the critical point are incorrect.
    \item In $d \geq 4$, MFT both gives the correct phase diagram and qualitative predictions of behaviour.
\end{itemize}

This is similar for other systems: Mean field theory gets the phase structure wrong for $d \leq d_L$ for some \textit{lower critical dimension} and mean field theory is correct for $d > d_{c}$, for some \textit{upper critical dimension}. For Ising one has $d_c = 1$ and $d_L = 4$. For $d_c < d < d_L$ there is interesting disagreement.

\subsection{Critical Exponents}


The qualitative predictions that are often disagreeing in MFT are \textit{critical exponents}. Near the critical point, MFT predicts if $B = 0$,
\begin{align}
    \begin{cases}
        m \sim  \left( T_C - T \right)^{\beta}, & \beta = \frac{1}{2}, \text{~as $T \to T_C^{-}$,} \\
        c \sim C_{\pm} \left| T - T_C \right|^{\alpha}, & \alpha = 0, \text{~as $T \to T_C^{+}$,}\\
        \xi \sim  \left| T - T_C \right|^{-\xi}, & \xi = 1. 
    \end{cases}
\end{align}

If $T = T_C$ as $B \to 0$, $m \sim  B^{\frac{1}{\delta}}$ where $\delta = 3$. This behaviour is correct, and agrees with the exact solution of the Ising model in $d = 2$, but the values of $\alpha , \beta , \gamma, \delta$ are \textbf{not}.

\begin{table}[h]
    \centering
    \caption{Critical Exponents}
    \label{tab:ce}
    \begin{tabular}{c|ccc}
     & MFT & $d= 2$ & $d = 3$ \\
     \midrule
        $\alpha$ & 0 & 0 & 0.1101 \\
        $\beta$ & $\frac{1}{2}$ & $\frac{1}{8}$ & $0.3264$ \\
        $\gamma$ & $1$ & $\frac{7}{4}$ & 0.3264 \\
        $\delta$ & 3 & 15 & 4.7898
    \end{tabular}
\end{table}

The $d = 2$ data here comes from the exact solution to the Ising model and the $d = 3$ data come from simulations as there is no exact solution known.

There is a notion of universality here, in that for any normal material we see a similar Liquid-gas phase transition. Namely a line of first order phase transitions ending at a critical point with a second order phase transition.

The external magnetic field is instead the pressure $B \leftrightarrow p$  and our order parameter, which was magnetisation, is now $m \leftrightarrow v \equiv \frac{V}{N}$.

Similar to before, we have
\begin{align}
    \begin{cases}
        v_{\text{gas}} - v_{\text{liquid}} \sim  \left( T_C - T \right)^{\beta}, & \beta = \frac{1}{2}, \text{~as $T \to T_C^{-}$ and $p = p_C$,}\\
        v_{\text{gas}} - v_{\text{liquid}} \sim \left( p - p_C \right)^{\frac{1}{2}}, & \delta = 3, \text{~as $p \to p_C$ and at $T = T_C$.}
    \end{cases}
\end{align}

\subsection{Isothermal compressibility}

One has
\begin{align}
    \kappa = \frac{1}{v} \left( \pdv{v}{p} \right)_{T} \sim \frac{1}{\left| T - T_C \right|^{\xi}} 
,\end{align}
where $\xi = 1$. The heat capacity is
\begin{align}
    C_V \sim  C_{\pm} \left| T - T_C \right|^{-\alpha}
,\end{align}
where $\alpha = 0$. The predictions for $\alpha, \beta, \xi, \delta$ are the same as the MFT for the Ising model.

This is surprising and experiment verifies that the true values of these critical exponents are the same as the critical values for $d = 3$ Ising theory.

This is an example of \textbf{universality}: different physical systems can exhibit the same behaviour at the critical point. This suggests that the microscopic physics is unimportant at a critical point. Systems governed by the same critical point belong to the same \textbf{universality class}.

\subsection{Landau-Ginzburg theory}

Therefore, if we can describe one such theory in a universality class, we can describe them all. Therefore our aim becomes: find one model that correctly describes the \textit{long-distance} physics near the critical point and use it to compute the critical exponents for all theories in the same universality class.

Landau-Ginzburg theory, also called \textit{Landau theory} generalizes MFT to allow for spatial variations in $m$. $M\left( \vb{x} \right) $ is now a field.

The field $M\left( \vb{x} \right) $ is produced from a microscopic model by \textbf{course-graining}, where we average over scales small compared to the length scales we are interested in but large compared to the microscopic physics.

In the Ising model, for example, one can course grain the lattice over boxes of side length $a$ each with $N' \ll N$ sites. We define
\begin{align}
    M\left( \vb{x} \right) = \frac{1}{N'} \sum \text{~spins in box with center $\vb{x}$}
.\end{align}

Take $N' \gg 1$ so disorderedness of $M\left( \vb{x} \right) $ can be ignored, $M\left( \vb{x} \right)  \in \left[ -1,1 \right] $. We also assume $a \ll \xi$, the length scale over which physics varies (i.e. Temperature etc.).

We then treat $M \left( \vb{x} \right) $ as a smooth function that doesn't rely on scales less than $a$. We then proceed as before.

The partition function can be written
\begin{align}
    Z &= \sum_{M \left( \vb{x} \right) } \sum_{\{s_{i}\}  \mid M\left( \vb{x} \right) } e^{-\beta E \left[ s \right] } \\
      &=  \sum_{M \left( \vb{x} \right) }^{} e^{-\beta F \left[ M \left( \vb{x} \right)  \right] }
  ,\end{align}
  where $F \left[ M \left( \vb{x} \right)  \right] $ is a \textit{functional} called the \textbf{Landau-Ginzburg free energy}.

\begin{definition}
    We convert this sum over all possible functions to a \textbf{path integral}, written and defined such that
    \begin{align}
        Z = \int \mathcal{D}\left[ M\left( \vb{x} \right) \right]  e^{-\beta F \left[ M \left( \vb{x} \right)  \right] }
    .\end{align}
    This is a \textit{functional integral} which sums over all possible fields $M\left( \vb{x} \right) $ that don't vary on scales less than $a$.
\end{definition}








