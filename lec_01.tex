\lecture{1}{11/10/2024}{Introduction}
% Harvey Reall
% Office Hours are Friday 2-3pm B2.09
% Goldenfeld
% Kardar

\subsection*{Motivation}

This course concerns itself with \textbf{universality}: the idea that different physical systems can exhibit the same behaviour.

For example, a liquid gas system has a critical temperature, below which a first order phase transition can be found between liquid and gas. Above it there is no such distinction and one can only move continuously.

Experiments suggest that as a function of the temperature, the density is given by $\left| \rho^{\pm} - \rho_c \right| \propto \left|  T - T_C \right|^{\beta}$ with $\beta \approx 0.327$ for $T \approx T_C$.

Another example is ferromagnets for which one also has a \textbf{critical temperature} called the Curie temperature $T_C$.

For $T > T_C$, $M = 0$, and for $T \leq T_C$ $M \propto \left( T_C - T \right)^{\beta}$ where the critical exponent $\beta$ is as observed for gases.

In this course we will be studying the \textbf{classical statistical mechanics of fields}.

%\subsection{From Spins to Fields}

\subsection{The Ising Model}

The Ising model is a simple model for a magnet. In $d$ spatial dimensions, consider a lattice with $N$ sites.

On the $i$th site we have a ``spin'' $S_{i} \in \{-1,1\} $. The configuration $\{s_{i}\} $ has energy 
\begin{align}
    E = - B \sum_{i}^{s_{i}} - J \sum_{\left<ij \right>}^{} s_{i} s_{j}
,\end{align}
where $B$ represents an external magnetic field and $J$ represents an interaction strength. Naturally, one should ask how the physics depends on $B$, $J$ and the temperature $T$.

For:
\begin{itemize}
    \item $J > 0$, the spins prefer to be aligned: $\up \up$ or $\down \down$. This is a \textbf{ferromagnet}.
    \item $J < 0$, the spins prefer to be anti-aligned: $\up \down$ or $\down \up$. This is an \textbf{anti-ferromagnet}.

        We will assume and fix $J > 0$.

    \item For $B > 0$ spins prefer $\up$
    \item For $B < 0$ spins prefer $\down$


    \item At low temperature, the system wants to minimise energy. This leads to an ordered state where all spins will align due to the lower energy preferenced state.
    \item At high temperature, entropy dominates which leads to a disordered state.
\end{itemize}

Recall that in the \textbf{canonical ensemble}, the probability of a given configuration $\{s_{i}\} $ is
\begin{align}
    p \left[ s_{i}  \right] = \frac{1}{Z} e^{-\beta E \left[ s_{i} \right] }
,\end{align}
where $\beta = \frac{1}{T}$, $k_B = 1$ and $Z$ is the \textbf{partition function}.

\begin{definition}
    The \textbf{partition function} is given by
    \begin{align}
        Z\left( T,B \right) = \sum_{\{s_{i}\} }^{} e^{-\beta E \left[ s_{i}  \right] }
    .\end{align}
\end{definition}

\begin{definition}
    The \textbf{thermodynamic free energy} $F_{\text{thermo}}\left( T,B \right) = \left<E \right> - TS = -T \log Z $ 
\end{definition}

\begin{definition}
    The \textbf{magnetisation} is 
    \begin{align}
        m = \frac{1}{N} \left<\sum_{n=1}^{^{N}}  s_{i} \right> \in \left[ -1,1 \right] 
    .\end{align}
\end{definition}

\begin{note}
    The average $\left< X \right>$ is an average over the probability distribution $p\left[  s_{i}\right] $ at a fixed $T$.
\end{note}

Magnetisation is an \textbf{order parameter} as it can distinguish between ordered phases ($m \neq 0$) and disordered phases where $m \approx 0$. We can expand the expectation value in $m$ such that
\begin{align}
    m = \sum_{\{s_{i}\} }^{} \frac{1}{Z} e^{-\beta \left[ E_{i} \right] } \frac{1}{N} \sum_{i}^{} s_{i} = \frac{1}{N\beta} \pdv{\beta} \log Z
.\end{align}

Therefore we want to compute the partition function as it will allow us to derive these observables.

This is easy in $d = 1$. This is hard in $d = 2$ and there is no general solution for generic lattices. For a square lattice with $B=0$ it is possible and was famously solved by Onsager (winning him some famous prize). $d \geq 3$ are intractable analytically.

Our aim is to approximate in a way that correctly captures the long-distance behaviour.

We can define $m$ for any $\{s_{i}\} $ by $m = \frac{1}{N} \sum_{}^{} s_{i}$, now no longer taking a statistical average. We can then write the partition function as
\begin{align}
    Z = \sum_{m} \sum_{\{s_{i}\} \mid m}  e^{-\beta E \left[ s_{i} \right] } =: \sum_{m}^{} e^{-\beta F \left( m \right) }
.\end{align}

The spacing in allowed values of $m$ is $\frac{2}{N}$. For $N \gg 1$ we can approximate $m$ as continuous however and thus write the partition function as
\begin{align}
    Z \approx \frac{N}{2} \int_{-1}^{1} \dd{m} e^{-\beta F\left( m \right) }
.\end{align}

We call $F\left( m \right) $ the \textbf{effective free energy} which depends on $T,B$ and critically, $m$. This contains more information than $F_{\text{thermo}}$.

Let $f\left( m \right) = \frac{1}{N}F\left( m \right)$ and thus
\begin{align}
    Z \propto \int_{-1}^{1} \dd{m} e^{-\beta N f\left( m \right) }
.\end{align}

For $N$ large, $\beta f\left( m \right) \sim  \mathcal{O}\left( 1 \right) $ as it is an \textbf{intensive} property (doesn't scale with the system). Performing a saddle point approximation, we can replace $f\left( m \right) $ by it's minimum where
\begin{align}
    \pdv{f}{m} \bigg|_{m=m_{\text{min}}} = 0
,\end{align}
which gives us a partition function of
\begin{align}
    Z \propto e^{-\beta N f\left( M_{\text{min}} \right) }
,\end{align}
which gives us a thermodynamic free energy of
\begin{align}
    F_{\text{thermo}}\left( T,B \right) \approx F\left( M_{\text{min}}\left( T,B \right) , T,B \right) 
.\end{align}

Even with the saddle point approximation, computing $F\left( m \right) $ is hard. One such approach is to use the \textbf{mean field approximation} where we replace each spin by the average of the field, $s_{i} \to m$. This gives
\begin{align}
    E = - \beta \sum_{_{i}}^{} m - J \sum_{\left<ij \right>}^{} m^2 = - B N m - \frac{1}{2} N J q m^2
,\end{align}
where $q$ is the number of nearest neighbours. $q = 2d+2$ for a square lattice in $d$ dimensions. 

Therefore in the mean field approximation, the partition function becomes
\begin{align}
    Z \approx \sum_{m}^{} \Omega \left( m \right) e^{-\beta E\left[ n \right] } 
,\end{align}
where $\Omega\left( m \right) $ is the number of configurations with $\frac{1}{N} \sum s_{i} = m$.
